%!TEX root = ../main.tex

%------------------------------
%  Do not add notation macros to this file 
%  use notation.tex instead
%-------------------------------

%----- Full version switch --------------------------------------------------------
\newboolean{submission} 
\newboolean{finalversion} 
\newboolean{fullversion} 

\newcommand{\issubmission}{%
\setboolean{submission}{true}
\setboolean{finalversion}{false} 
\setboolean{fullversion}{false} 
}

\newcommand{\isfinalversion}{%
\setboolean{submission}{false}
\setboolean{finalversion}{true} 
\setboolean{fullversion}{false} 
}

\newcommand{\isfullversion}{%
\setboolean{submission}{false}
\setboolean{finalversion}{false} 
\setboolean{fullversion}{true} 
}


\newcommand{\submission}[1]{\ifthenelse{\boolean{submission}}{#1}{}}
\newcommand{\finalversion}[1]{\ifthenelse{\boolean{finalversion}}{#1}{}}
\newcommand{\fullversion}[1]{\ifthenelse{\boolean{fullversion}}{#1}{}}
% #1 for submission #2 for finalversion #3 for fullversion
\newcommand{\whichversion}[3]{\ifthenelse{\boolean{submission}}{#1}{\ifthenelse{\boolean{finalversion}}{#2}{#3}}}

%-------------------------------------------------------------------------------
\makeatletter %Enable pdf table of contents in the full version
\fullversion{
	\renewcommand*\l@section[2]
	{%
		\ifnum \c@tocdepth >\z@
		\addpenalty \@secpenalty
		\addvspace {1.0em \@plus \p@ }%
		\setlength \@tempdima {1.5em}%
		\begingroup
		\parindent \z@
		\rightskip \@pnumwidth
		\parfillskip -\@pnumwidth
		\leavevmode \bfseries
		\advance \leftskip \@tempdima
		\hskip -\leftskip
		#1\nobreak
		\hfil
		\nobreak
		\hb@xt@ \@pnumwidth {\hss #2\kern -\p@ \kern \p@ }%
		\par
		\endgroup
		\fi
	}
}
\makeatother
%-------------------------------------------------------------------------------


%----- Itemize  ----------------------------------------------------------------
\newcommand{\bi}{\begin{itemize}}
\newcommand{\ei}{\end{itemize}}
\newcommand{\be}{\begin{enumerate}}
%\newcommand{\ee}{\end{enumerate}}
\newcommand{\bein}{\begin{enumerate*}}
\newcommand{\eein}{\end{enumerate*}}
\setlist{itemsep=2pt,topsep=2pt,parsep=2pt,partopsep=2pt} % spacing

%----- Table  ------------------------------------------------------------------
%\rowcolors{2}{gray!25}{white}

%----- Theorems ----------------------------------------------------------------

%LNCS related theorem environments
%\spnewtheorem*{theorem*}{Theorem}{\bfseries}{\itshape}
%\spnewtheorem*{theoremnn}{Theorem}{\bfseries}{\itshape}
%\spnewtheorem*{lemma*}{Lemma}{\bfseries}{\itshape}
%\spnewtheorem*{definition*}{Definition}{\bfseries}{\itshape}
%\spnewtheorem{fancyclaim}{Claim}{\bfseries}{\itshape}
%\crefname{fancyclaim}{Claim}{Claims}
%%This only works with LNCS!
%\makeatletter
%\spnewtheorem{repeatthm@}{Theorem}{\bfseries}{\itshape}
%\newenvironment{repeatthm}[1]{%
%    \def\therepeatthm@{\ref{#1}}
%    \repeatthm@
%}
%{\endrepeatthm@}
%\spnewtheorem{repeatlem@}{Lemma}{\bfseries}{\itshape}
%\newenvironment{repeatlem}[1]{%
%    \def\therepeatlem@{\ref{#1}}
%    \repeatlem@
%}
%{\endrepeatlem@}
%\spnewtheorem{repeatcor@}{Corollary}{\bfseries}{\itshape}
%\newenvironment{repeatcor}[1]{%
%    \def\therepeatcor@{\ref{#1}}
%    \repeatcor@
%}
%{\endrepeatcor@}
%\makeatother
%Non-LNCS theorem environments
%\newtheorem{theorem}{Theorem}
%\newtheorem{lemma}{Lemma}
%\newtheorem{corollary}{Corollary}
%\newtheorem{definition}{Definition}
%\newtheorem{assumption}{Assumption}
%\newtheorem{fancyclaim}{Claim}
%\crefname{fancyclaim}{Claim}{Claims}
%\newtheorem{remark}{Remark}
%\newtheorem{claim}{Claim}

%This might work with amsthm
%\makeatletter
%\newtheorem{repeatthm@}{Theorem}{\bfseries}{\itshape}
%\newenvironment{repeatthm}[1]{%
%    \def\therepeatthm@{\ref{#1}}
%    \repeatthm@
%}
%{\endrepeatthm@}
%\makeatother

\newenvironment{claimproof}[1]{\par\noindent\underline{Proof}:\space#1}{\hfill $\blacksquare$}


%-------------------------------------------------------------------------------
%  Magic Stuff below
%-------------------------------------------------------------------------------

%------ Quote ------------------------------------------------------------------
\renewcommand{\quote}{\list{}{\rightmargin=\leftmargin\topsep=0pt}\item\relax}

%------ Subsection and Paragraph -----------------------------------------------

% Saving space in case of deadlines

%\makeatletter
%\renewcommand{\section}{\abovedisplayskip 3\p@ \@plus3\p@ \@minus1\p@%
%                      \belowdisplayskip 5\p@ \@plus3\p@ \@minus1\p@%
%                      \abovedisplayshortskip 0pt \@plus2\p@%
%                      \belowdisplayshortskip 0pt \@plus2\p@ \@minus0\p@%
%                      \@startsection{section}{1}{\z@}%
%                       {-10\p@ \@plus -4\p@ \@minus -4\p@}%
%                       {6\p@ \@plus 4\p@ \@minus 4\p@}%
%                       {\normalfont\large\bfseries\boldmath
%                        \rightskip=\z@ \@plus 8em\pretolerance=10000 }}
%\renewcommand{\subsection}{\@startsection{subsection}{2}{\z@}%
%                      {-6\p@ \@plus -4\p@ \@minus -4\p@}%
%                      {2\p@ \@plus 2\p@ \@minus 2\p@}%
%                      {\normalfont\normalsize\bfseries\boldmath
%                       \rightskip=\z@ \@plus 8em\pretolerance=10000 }}
%\renewcommand{\subsubsection}{\@startsection{paragraph}{4}{\z@}%
%                      {-8\p@ \@plus -4\p@ \@minus -4\p@}%
%                      {-5\p@ \@plus -0.22em \@minus -0.1em}%
%                      {\normalfont\normalsize\bfseries\boldmath
%                      }}
\renewcommand{\paragraph}[1]{\smallskip\noindent{\bf #1}}
%\renewcommand{\paragraph}{\@startsection{paragraph}{4}{\z@}%
%                      {-8\p@ \@plus -4\p@ \@minus -4\p@}%
%                      {-5\p@ \@plus -0.22em \@minus -0.1em}%
%                      {\normalfont\normalsize\bf
%                     }}
%\makeatother
\newcommand{\stealspace}{\vspace{-5pt}}

%----- Comments ----------------------------------------------------------------
\RequirePackage{totcount}
\RequirePackage{color}
\RequirePackage[colorinlistoftodos]{todonotes}

\newtotcounter{notecount}
\newcommand{\notewarning}{%
\ifnum\totvalue{notecount}>0%
 \vspace{1ex}
\begin{center}
 \begin{tikzpicture}[baseline=(A.south)]
    \node (A) [] at (0,0){};
    \node [rounded corners=1pt,rectangle, draw=red, fill=red!20,text=black](B) at (0.1ex,0ex){
        \Large \raggedright {\bf Warning:} There are still some notes left!
    };
 \end{tikzpicture}
\end{center}
 \vspace{1ex}
\fi
}
\makeatletter
\def\myaddcontentsline#1#2#3{%
  \addtocontents{#1}{\protect\contentsline{#2}{#3}{Section \thesubsection\ at p. \thepage}{}}}
\renewcommand{\@todonotes@addElementToListOfTodos}{%
    \if@todonotes@colorinlistoftodos%
        \myaddcontentsline{tdo}{todo}{{%
            \colorbox{\@todonotes@currentbackgroundcolor}%
                {\textcolor{\@todonotes@currentbackgroundcolor}{o}}%
            \ \@todonotes@caption}}%
    \else%
        \myaddcontentsline{tdo}{todo}{{\@todonotes@caption}}%
   \fi}%
\newcommand*\mylistoftodos{%
  \begingroup
       \setbox\@tempboxa\hbox{Section 9.9 at p. 99}%
       \renewcommand*\@tocrmarg{\the\wd\@tempboxa}%
       \renewcommand*\@pnumwidth{\the\wd\@tempboxa}%
       \listoftodos%
  \endgroup
}
\makeatother

%------------------------------------
% COLOR
%------------------------------------
\definecolor{codegreen}{rgb}{0,0.6,0}
\definecolor{codegray}{rgb}{0.5,0.5,0.5}
\definecolor{codepurple}{rgb}{0.58,0,0.82}
\definecolor{backcolour}{rgb}{0.95,0.95,0.92}
\definecolor{mDarkBrown}{HTML}{604c38}
\definecolor{mDarkTeal}{HTML}{23373b}
\definecolor{mLightBrown}{HTML}{EB811B}
\definecolor{mLightGreen}{HTML}{14B03D}
\definecolor{lightgreen}{rgb}{0.86, 0.93, 0.78}
\definecolor{textgreen}{rgb}{0.0, 0.5, 0.0}
\definecolor{bordergreen}{rgb}{0.55, 0.76, 0.74}
\definecolor{lightblue}{rgb}{0.70, 0.90, 0.99}
\definecolor{borderblue}{rgb}{0.01, 0.66, 0.96}
\definecolor{lightamber}{rgb}{1, 0.93, 0.70}
\definecolor{borderamber}{rgb}{1, 0.76, 0.03}
\definecolor{lightcolor4}{rgb}{ 0.93, 0.70, 1}
\definecolor{bordercolor4}{rgb}{0.76, 0.03, 1}
\definecolor{lightcolor5}{rgb}{0.78,0.86,0.93}
\definecolor{bordercolor5}{rgb}{0.74,0.55,0.76}
%Even more Colors
\definecolor{blueinfo}{RGB}{64, 112, 173}%{rgb}{0.25,0.44,0.68} %blue
\definecolor{greeninfo}{RGB}{148, 176, 54}%{rgb}{0.58,0.69,0.21} %green
\definecolor{yellowinfo}{RGB}{240, 189, 82}%{rgb}{0.94,0.74,0.32} %yellow
\definecolor{redinfo}{RGB}{194, 77, 59}%{rgb}{0.76,0.30,0.23} %red
\definecolor{devpurple}{RGB}{125, 58, 193}%dark orchid
\definecolor{implgray}{RGB}{193, 193, 193}
\definecolor{greencomment}{RGB}{118,191,105}
\definecolor{bluecomment}{RGB}{105,166,191}
\definecolor{yellowcomment}{RGB}{239,200,115}
\definecolor{redcomment}{RGB}{194, 77, 59}

%\hypersetup{linkcolor=}
\newcommand{\citeColored}[2]{{\hypersetup{citecolor=#1}\cite{#2}}}
\newcommand{\citeGreen}[1]{\citeColored{codegreen}{#1}}
\newcommand{\citePurple}[1]{\citeColored{codepurple}{#1}}

\newcommand{\green}[1]{\textcolor{mLightGreen}{#1}}
\newcommand{\dgreen}[1]{\textcolor{mDarkTeal}{#1}}
\newcommand{\orange}[1]{\textcolor{mLightBrown}{#1}}
\newcommand{\dorange}[1]{\textcolor{mDarkBrown}{#1}}
\newcommand{\purple}[1]{\textcolor{codepurple}{#1}}

\newcommand{\executor}{evaluator }
\newcommand{\Executor}{Evaluator }
\newcommand{\pval}{P_{e}}

\ifDraft
\newcommand{\girisha}[1]{\textcolor{blue}{[{\footnotesize {\bf Girisha:} { {#1}}}]}}
\newcommand{\bhavana}[1]{\textcolor{red}{[{\footnotesize {\bf Bhavana:} { {#1}}}]}}
\newcommand{\chaya}[1]{\textcolor{magenta}{[{\footnotesize {\bf Chaya:} { {#1}}}]}}
\newcommand{\shreyas}[1]{\textcolor{green}{[{\footnotesize {\bf Shreyas:} { {#1}}}]}}

\else
\newcommand{\girisha}[1]{}
\newcommand{\bhavana}[1]{}
\newcommand{\chaya}[1]{}
\newcommand{\shreyas}[1]{}
\fi

%Others
%\newcommand{\replace}[2]{[\textbf{REPLACE}: \textcolor{red}{#1} \textcolor{blue}{#2}]}
%\newcommand{\remove}[1]{[\textbf{REMOVE}: \textcolor{red}{#1}]}
\newcommand{\alert}[1]{\textcolor{red}{#1}}


%----- Algorithm Environment ---------------------------------------------------
%Header for Algorithms/Functionalities
\newcommand{\algoHead}[1]{\vspace{0.2em} \underline{\textbf{#1}} \vspace{0.3em}}
\newcommand{\algoHeadExt}[2]{\vspace{0.2em} \underline{\textbf{#1} #2} \vspace{0.3em}}

%Multiline Algo-States
\makeatletter
\algnewcommand{\ExtendedState}[1]{\State
\parbox[t]{\dimexpr\linewidth-\ALG@thistlm}{\hangindent=\algorithmicindent\strut\hangafter=3#1\strut}}
\makeatother

%Algorithms States
\algnewcommand\algorithmicinput{\textbf{Input:}}
\algnewcommand\Input{\item[\algorithmicinput]}
\renewcommand{\algorithmicensure}{\textbf{Output:}}
%Algo Comments
\algrenewcommand{\algorithmiccomment}[1]{{\color{gray}// #1}}
%Inline ifs
\algnewcommand{\IIf}[1]{\State\algorithmicif\ #1\ \algorithmicthen}
\algnewcommand{\EndIIf}{\unskip\ \algorithmicend\ \algorithmicif}

%----- Box Environment -----------------------------------------
\RequirePackage{color}
\RequirePackage[most]{tcolorbox}%with most option
\RequirePackage{totcount}
%fonteawesome if possible

%fonteawesome if possible
\IfFileExists{fontawesome.sty}{%
\RequirePackage{fontawesome}
}{%Icons Fallback if one cannot use fontawesome
\providecommand{\faInfoCircle}{\hspace{3pt}\textbf{i}}
\providecommand{\faExclamationCircle}{\hspace{3pt}\textbf{!}}
\providecommand{\faExclamationTriangle}{\hspace{3pt}\textbf{!}}
\providecommand{\faCogs}{\hspace{2pt}\textbf{D}}
\providecommand{\faComment}{}
\providecommand{\faHSquare}{\hspace{2pt}\textbf{H}}
} 
%----- Protocol Boxes
% Removed 'center' from the below. Seems to not make any difference
\newtcolorbox[auto counter]{titlebox}[6][]{
    enhanced,
    colframe=black,
    colback=white,
    boxrule={#4},
    arc={#3},
    auto outer arc,%
 	%breakable,
    pad at break*=0pt,
    vfill before first,
    before={\par\medskip\noindent},
    after={\par\medskip},
    top=12pt, left=4pt, enlarge top by=6pt,%enlarge bottom by=7pt,%
    title={\rule[-.3\baselineskip]{0pt}{\baselineskip}\normalsize\sffamily\bfseries #2}, 
    varwidth boxed title*=-10pt, 
   attach boxed title to top left={yshift=-10pt,xshift=10pt}, 
   coltitle=black,
    boxed title style={colback=white,boxrule={#6},arc={#5},auto outer arc},
    #1
}

%\newtcolorbox[auto counter,crefname={Protocol}{Protocols}]{protbox}[6][]{
%    enhanced,
%    colframe=black,
%    colback=white,
%    boxrule={#4},
%    arc={#3},
%    auto outer arc,%
% 	%breakable,
%    pad at break*=0pt,
%    vfill before first,
%    before={\par\medskip\noindent},
%    after={\par\medskip},
%    top=12pt, left=4pt, enlarge top by=6pt,%enlarge bottom by=7pt,%
%    title={\rule[-.3\baselineskip]{0pt}{\baselineskip}\normalsize\sffamily\bfseries #2}, 
%    varwidth boxed title*=-10pt, 
%   attach boxed title to top left={yshift=-10pt,xshift=10pt}, 
%   coltitle=black,
%    boxed title style={colback=white,boxrule={#6},arc={#5},auto outer arc},
%    #1
%}

\newenvironment{systembox}[1]
{\vspace{\baselineskip}\begin{titlebox}{Functionality \normalfont #1}{2.5pt}{1pt}{3.5pt}{1pt}}
{\end{titlebox}}

\newenvironment{protocolbox}[2][]
{\begin{protbox}[#1]{Protocol~\thetcbcounter: \normalfont #2}{0.5pt}{0.5pt}{1pt}{0.75pt}}
{\end{protbox}}

 \newenvironment{processbox}[1]
{\begin{titlebox}{Process \normalfont #1}{0.5pt}{0.5pt}{1pt}{0.75pt}}
{\end{titlebox}}

\newenvironment{algobox}[1]
{\begin{titlebox}{Algorithm \normalfont #1}{0.5pt}{0.5pt}{1pt}{0.75pt}}
{\end{titlebox}}
%{\begin{titlebox}{Algorithm \normalfont #1}{commonbox}{normal}}
%{\end{titlebox}}

\newenvironment{funcbox}[1]
{\begin{titlebox}{Function \normalfont #1}{0.5pt}{0.5pt}{1pt}{0.75pt}}
{\end{titlebox}}

\newenvironment{gamebox}[1]
{\begin{titlebox}{Game \normalfont #1}{0.5pt}{0.5pt}{1pt}{0.75pt}}
{\end{titlebox}}

\newenvironment{oraclebox}[1]
{\begin{titlebox}{Oracle \normalfont #1}{0.5pt}{0.5pt}{1pt}{0.75pt}}
{\end{titlebox}}

%----- Allow subsubsection ----------------------------
\setcounter{secnumdepth}{3}

%----- Reference magic ---------------------------------------------------------
%Enable reference of descriptions
\makeatletter
\let\orgdescriptionlabel\descriptionlabel
\renewcommand*{\descriptionlabel}[1]{%
  \let\orglabel\label
  \let\label\@gobble
  \phantomsection
  \edef\@currentlabel{#1}%
  %\edef\@currentlabelname{#1}%
  \let\label\orglabel
  \orgdescriptionlabel{#1}%
}
\makeatother

%----- 2P protocol ---------------------------------------------------------
\usepackage{tabularx}
\usepackage{array}
\usepackage{arydshln}
%--Define protocol environment
\newcommand{\PreserveBackslash}[1]{\let\temp=\\#1\let\\=\temp}
\newcolumntype{C}[1]{>{\PreserveBackslash\centering}b{#1}}
\newcolumntype{R}[1]{>{\PreserveBackslash\raggedleft}b{#1}}
\newcolumntype{L}[1]{>{\PreserveBackslash\raggedright}b{#1}}
\newenvironment{twopartyprotocol}{%
\renewcommand{\arraystretch}{1.5}
\bigskip\par
\tabularx{\textwidth}{L{0.3\textwidth} C{0.32\textwidth} R{0.3\textwidth}}
}{%
\endtabularx %cannot use \begin{tabularx}\end{tabularx} within environments
\bigskip
}
\newcommand{\TPPtopline}[3]{#1&#2&#3\\\hline&&\\[-\medskipamount]}%Header
%\newcommand{\TPPfirstline}[3]{\rule[2ex]{0pt}{3ex}#1&#2&#3\\}%comes with spacing on top
\newcommand{\TPPline}[3]{#1&#2&#3\\}
\newlength{\TPParrow}
\settowidth{\TPParrow}{\scriptsize$a really long message over the arrow$} %this defines the (min) length of arrows
\newcommand*{\TPPrightarrow}[1]{\xmath{\xrightarrow{\mathmakebox[\TPParrow]{#1}}}} %use substack for multi line
\newcommand*{\TPPleftarrow}[1]{\xmath{\xleftarrow{\mathmakebox[\TPParrow]{#1}}}}

% Macros specific to Second Price auctions
\newcommand{\cb}{\mathsf{cb}} % For choice bit 

\newcommand{\bcot}{\mathsf{bcot}} % For bit code shared using OT
\newcommand{\bcbb}{\mathsf{bcbb}} % For bit code written to BB 
\newcommand{\zo}{{0 \rightarrow 1}}
\newcommand{\oz}{{1 \rightarrow 0}}
\newcommand{\zz}{{0 \rightarrow 0}}
\newcommand{\oo}{{1 \rightarrow 1}}
\newcommand{\cals}{\mathcal{S}}
\newcommand{\calc}{\mathcal{C}}
\newcommand{\calf}{\mathcal{F}}
\newcommand{\calr}{\mathcal{R}}
\newcommand{\calh}{\mathcal{H}}
\newcommand{\sfm}{\mathsf{M}}
\newcommand{\sfr}{\mathsf{R}}
\newcommand{\sfc}{\mathsf{C}}
\newcommand{\cbits}{\mathsf{cbits}}
\newcommand{\crh}{\mathsf{hash}}
\newcommand{\com}{\mathsf{Com}}
\newcommand{\zq}{\mathbb{Z}_q}
\newcommand{\grp}{\mathbb{G}}
\newcommand{\otr}{\mathsf{otr}}
\newcommand{\ots}{\mathsf{ots}}
\newcommand{\strspc}{\Phi}
\newcommand{\view}{\mathsf{View}}
\newcommand{\pay}{\overset{\small pay} {\longleftarrow}}
\newcommand{\smpl}{\overset{\small \$} {\longleftarrow}}
\newcommand{\bbwrite}{\overset{\small write} {\longleftarrow}} 
\newcommand{\coll}{\mathsf{coll}_{\calc}}
\newcommand{\calv}{\mathcal{V}}
\newcommand{\val}{\calv}